\documentclass{article}
\usepackage[utf8]{inputenc}
\usepackage{epigraph}
\usepackage{listings}
\usepackage[spanish]{babel}
\usepackage{color}
\usepackage{graphicx}
\usepackage{tikz}
\usepackage[letterpaper, total={7in, 10in}]{geometry}
\title{\centerline{\textbf{ALGORITMOS: AYUDANTÍA 5}}}
\author{\centerline{Ayudante: Yerko Ortiz}}
\date{}
\definecolor{mygreen}{rgb}{0,0.6,0}
\definecolor{mygray}{rgb}{0.5,0.5,0.5}
\definecolor{mymauve}{rgb}{0.58,0,0.82}

\lstset{ %
  backgroundcolor=\color{white},   % choose the background color
  basicstyle=\footnotesize,        % size of fonts used for the code
  breaklines=true,                 % automatic line breaking only at whitespace
  captionpos=b,                    % sets the caption-position to bottom
  commentstyle=\color{mygreen},    % comment style
  escapeinside={\%*}{*)},          % if you want to add LaTeX within your code
  keywordstyle=\color{blue},       % keyword style
  stringstyle=\color{mymauve},     % string literal style
  showstringspaces=false
}
\begin{document}

\maketitle
\begin{flushleft}
\textbf{Objetivo de la ayudantía: Reforzar conceptos referentes a algoritmos de ordenamiento.}
\vspace{1cm}
\hrule
\vspace{1cm}
\section*{\centerline{Problema}}
\subsection*{Entrada}
\begin{itemize}
    \item 
\end{itemize}
\subsection*{Dominio}
\begin{itemize}
        \item
\end{itemize}
\subsection*{Salida}
    \item
\subsection*{Caso de prueba}
\begin{itemize}
    \item Entrada: \newline
    \item Salida: \newline
\end{itemize}
\hrule 
\vspace{1cm}
\section*{\centerline{Compilador mental}}
Dada cierta entrada, ejecute y compile el siguiente algoritmo usando lápiz, papel e imaginación; describa la complejidad del algoritmo usando la notación \textbf{O(f(n))}.
\begin{lstlisting}[language=Java, frame=ltrb]
    static void function(int n) {
        return n;
    } 
\end{lstlisting}
\begin{itemize}
    \item \textbf{Entrada:} \newline
    \item \textbf{Salida:} \newline
    \item \textbf{Complejidad:} \newline
    \vspace{2cm}
\end{itemize}
\hrule
\vspace{0.5cm}
\centerline{\textbf{Gracias por su atención!}}
\epigraph{\textit{“Epígrafe”}}{Alguien}
\end{flushleft}
\end{document}


